\section{Blindness}
Visual impairment affects over 2 million people in the UK, including \num{340,000} people who are registered blind or partially sighted \cite{national_health_system_england_blindness_2017}.
The most common causes in the UK are age-related macular degeneration, glaucoma, cataracts, and diabetic retinopathy \cite{fight_for_sight_facts_nodate}.
Ultimately, 79\% of people with sight loss in the UK are aged 64 and over \cite{royal_national_institute_of_blind_people_key_nodate}.

Individuals can register as partially sighted or blind with the help of their ophthalmologist or general practioner (GP).
This gives access to the assistance one might need --- not just walking canes, guide dogs, audible systems, but also access to reduced fees and other monetary assistance.
The National Health System (NHS) works closely with the Royal National Institute of the Blind (RNIB) to help visually impaired people deal with their lifestyle needs; including providing braille classes, access to technology to help them, counselling to help them come to terms with their diagnosis and much more \cite{royal_national_institute_of_blind_people_key_nodate}.

Many technologies developed to assist people with visual impairments rely on hearing.
Some of these include talking watches and clocks, audio books and newspapers, smart-readers, and audio labellers \cite{royal_national_institute_of_blind_people_audio_nodate}. 
The reliance on audio means these devices are not suitable for the \num{400,000} people in the UK who have multi-sensory impairment, i.e. deafblindness \cite{royal_national_institute_of_blind_people_key_nodate}.
Deafblindness is also most common amongst an aging population, so it is crucial that devices to aid those with a visual impairment do not rely exclusively on hearing, and are reasonably accessible to the elderly \cite{sense_deafblindness_nodate,deafblind_uk_what_nodate,national_health_system_england_deafblindness_2017}.  

\section{Braille}
Louie Braille invented the tactile coded system in the 1820s after becoming blind aged 5.
He was inspired by the French Navy's system of ``night writing'' and developed a way to use 6 dots to display the alphabet, numbers, and music.
Today, braille is used by blind, partially sighted and deafblind people, giving them access to literacy, more freedom and independence \cite{sight_scotland_who_nodate}.
Braille can be taught to people at any age, so it does not matter if the blindness is congenital or acquired.
Teaching braille to children with some visual impairment is vital as it helps them to learn the concepts of grammar, spelling and punctuation, which is harder to learn via audio \cite{sight_scotland_what_nodate}.  

Braille is made up of six dots in two columns of three dots and different combinations of raised dots represent different letters and numbers.
Its dimensions are regulated by the UK Association for Accessible Formats (UKAAF). 
Uncontracted braille is a straight translation from text that includes full words and punctuation, resulting in long manuscripts.
Most books use a contracted braille, which includes abbreviations for common words and sounds, e.g. one character for common sounds such as ``sh'' and ``ing'' and single signs common short words including ``the'' and ``for''.
By using contracted braille, documents can become 25\% shorter and be quicker to read \cite{royal_national_institute_of_blind_people_braille_nodate,sense_braille_nodate}.