\section{Blindness}
In the UK there are 2 million people living with some kind of vision loss, including 340,000 people who are registered as blind or partially sighted.
Individuals can register as partially sighted or blind with the help of their ophthalmologist and GP. This will enable them to have access to the assistance they might need including help with additional costs and reduced fees.
The NHS works closely with the Royal National Institute of the Blind (RNIB) to help visually impaired people deal with their lifestyle needs; including providing braille classes, access to technology to help them, counselling to help them come to terms with their diagnosis and much more \cite{national_health_system_england_blindness_2017,royal_national_institute_of_blind_people_key_nodate}.
There are lots of tools available to help blind people go about day-to-day life such as walking canes, guide dogs and audible systems.  

Every 6 minutes in the UK someone is told that they are going blind \cite{fight_for_sight_facts_nodate}, and this can be due to many reasons and diseases.
There are over 100 eye conditions but the most common causes in the UK are age-related macular degeneration, glaucoma, cataracts, and diabetic retinopathy.
Cancer and trauma incidents can also lead to blindness.
79\% of people with sight loss in the UK are aged 64 and over \cite{royal_national_institute_of_blind_people_key_nodate}.  

Many technologies that have been developed to assist people with visual impairments, rely on the use of hearing.
Some of these include talking watches and clocks, audio books and newspapers, smart-readers, and audio labellers \cite{royal_national_institute_of_blind_people_audio_nodate}. 
These devices do not take into account the 400,000 people in the UK that live with deafblindness which is also referred to as multi-sensory loss.
People who are deafblind aren't necessarily totally deaf and/or blind but they will have both some sort of vision and hearing loss.
Deafblindness is particularly more common amongst the aging population, so it is therefore crucial that visual impairment devices do not totally rely on hearing \cite{sense_deafblindness_nodate,deafblind_uk_what_nodate,national_health_system_england_deafblindness_2017}.  

\section{Braille}
Louie Braille invented the tactile coded system in the 1820s after becoming blind aged 5.
He was inspired by the French Navy's system of ``night writing'' and developed a way to use 6 dots to display the alphabet, numbers, and music.
Braille is used nowadays by blind, partially sighted and deafblind people.
It gives these people access to literacy and gives them more freedom and independence \cite{sight_scotland_who_nodate}.  

Braille can be taught to people at any age, so it does not matter whether the individual was born without sight or lost it later in life.
Some blind people choose to not learn braille if they are able to rely on audible queues, but this is not a choice for those with deafblindness.
Teaching braille to children with some visual impairment is vital as it helps them to learn the concepts of grammar, spelling and punctuation, which is harder to learn via audio \cite{sight_scotland_what_nodate}.  

Braille can take a while to learn, but like with any language, the more you practice the easier it becomes.
Braille is made up of six dots in two rows and different combinations of raised dots represent different letters and numbers.
There are two grades of braille, 1 and 2. Grade 1, which is now known as uncontracted braille, is a straight translation from text and it includes full words and punctuation.
Contracted braille, formally grade 2, uses abbreviations for common words, one character for common sounds such as ``sh'' and ``ing'' and single signs common short words including ``the'' and ``for''.
By using contracted braille, documents can become 25\% shorter and they tend to be quicker to read \cite{royal_national_institute_of_blind_people_braille_nodate,sense_braille_nodate}.  