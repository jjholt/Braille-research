\section{Ethics}
To ensure our prototype is working well and effectively we will recruit participants to trial the device and provide feedback. These participants will be seeing, visually impaired or blind, but they will all have some knowledge of how to read braille. Including a trial in our research means that we have to make ethical considerations. The considerations include using and storing participants' personal data. We will handle all data in accordance with the data protection act of 2018 and will remove any identifiable data from anything being published. All personal data will be stored confidently and only accessed by members of VirtualEyes. 

The trial of the prototype will consist of the participant reading several characters using the device and then giving feedback to the team. By trialling the device it will enable VirtualEyes to improve the device before making further developments. The questions asked to the participants will give VirtualEyes quantitative and qualitative data. The ethics approval form, including the participant information sheet, the participant questionnaire  and the participant consent form is included in the appendix of this document. 
