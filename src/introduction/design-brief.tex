\section{The problem}
The design focuses on producing an affordable device to read and produce braille at home.
 

\subsection{Stakeholders}
There are a variety of stakeholders that will benefit from such a device.
Firstly, people with visual impairment.
The design introduces a new and convenient way to read text allowing them to become more independent.
Especially for individuals suffering from both hearing and visual impairment, this device provides a groundbreaking solution, since most existing reading devices rely on audio to convey the text to the reader.
The family and friends of the visually impaired also benefit as a result.
Since their loved ones become more self-reliant, they are relieved from part of their duties that may have involved assisting with reading.
It also provides them with a way to communicate in writing with their loved one without requiring the knowledge of braille.
At the same time, the design will provide small business with a chance to become more inclusive towards the visual impaired without requiring big financial investments.
Finally, organizations such as the Royal National Institute of the Blind (RNIB) also benefit as they are provided with more tools to assist their members and patients to assist them in a more effective communication and ultimately a more normal and independent life.
It's worth mentioning that since there is an existing industry for commercial braille embossing, who may try to inhibit the development and growth of such a braille printing / reading device.


\subsection{Objectives and criteria}
The main goal of this project is to produce an affordable device able to output braille text without requiring any input other than the original English text.
It goes without saying that to make a device affordable to the general public, the cost must be kept low.
To achieve that, cheap microcontrollers and manufacturing techniques are going to be utilized.
Since the vast majority of the device's users are going to be visually and or hearing handicapped, the device must be easily operated by such individuals.
This can be achieved by, for example, using big buttons with embossed braille text that describe their function.
Additionally, the input and output parts of the device must be easy to identify and operate with minimal error.
Additionally, because the device is designed for personal use, its size must be convenient to store/carry and of course safe to use.
This means that the device cannot have any sharp edges or exposed wires or moving parts that could hide potential hazard for a visually impaired user.
Enclosing the device in a case is crucial for the avoidance of injuries.
Finally, the device must require no special add-ons to operate.
This would go against the original idea of simplifying tasks for the visually impaired.
 

\subsection{Risks and constraints}
When trying to create such a device there is a variety of risks and constraints involved.
 Injuries can cause delays to the manufacture of the device but can easily be avoided by following the established security protocols for the machines that will be utilised.
This includes, for example, appropriate utilisation of gloves, helmet and other protective equipment and using low currents for the circuit.
The device must also be built in a way that minimises the risk of injury when operated by a visually and possibly hearing-impaired person.
When trying to build the device, the limited availability of the parts is a factor that cannot be reliably controlled and might cause setbacks.
For that reason, we are allowing some room for adjustments in the design if deemed necessary.
Since the device will utilize both software and hardware, their limitations may influence the outcome.
Factors affected may include the operation speed and its capability of performing complicated tasks such as image to text conversion.
Additionally, training will have to be provided to the visually impaired users to ensure the correct use of the device and provide instructions on how to recognize and troubleshoot simple operation issues (e.g.
paper jam in case of a printer).
Finally, it is worth mentioning that due to the restricted time available for manufacture, the creation of complicated devices might not be a realistic goal.
In such a case a simpler version of the device can be produced as a proof of concept.