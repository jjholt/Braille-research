\section{Failure mode and effects analysis}
VirtualEyes has used failure mode and effects analysis (FMEA) to assess any potential problems that will arise whilst building the device and problems with the device's parts. The risks have been classified based on the probability of occurrence(P), the seriousness of the error(S) and the likelihood that the defect reaches the costumer(D) on a scale from 1-5 with 1 being the lowest and 5 the highest value. The risk priority(R) of each situation was then calculate by multiplying those values (R = P*D*S).The analysis has been split into two sections: Process and Parts. 

The analysis of the parts of the device showed there are several potential hazards ranging from faulty batteries, with a risk priority score of 64 out of 75, to character pins not retracting properly with a score of 0/75. these can be mitigated by... 

Analysis of the processes to build the device highlighted several hazards that could occur. Potential risks included cutting one's hand whilst using wire cutters (6/75) and magnetic interference with surrounding pins (15/75).

More analytically, all the errors examined along with their corresponding risk priority score can be found in the appendix. (add reference of Table)
