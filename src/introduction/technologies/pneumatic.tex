Pneumatic mechanism is also a potential solution.
Its main advantage is that the power source is divorced from the actuator \cite{russomanno_design_2015}.
Power, in the form of pressurized air, is routed via small pipes and readily converted into the motion of a sliding pin using an elastic membrane.
This feature of separation of power source and functional module make pneumatic design promising in terms of forming a large-area dense array shape display.
However, one big challenge of pneumatic design is how to efficiently control large number of cells without sacrifice of portability \cite{russomanno_model-based_2017}.
The group proposed a fluidic logic system design shown in figure \ref{fig:pneumatic-schema} to solve this problem and the prototype showed positive results in the earlier tests \cite{russomanno_design_2015}.

\todo[inline]{``managed to release'' and ``still in pre-order''. I'd argue this needs to be more dismissive of their product and its promises as part of the reason why we wont want to touch fluids.}
Blitab managed to release their fluid-based product a few years ago but their product is still in pre-order state and never entered the market formally.
While very promising, we could not replicate it within our budgetary constraints.
\begin{figure}\centering
    \includegraphics[width=0.6\textwidth]{figures/pneumatic-schema.png}
\caption{Cross-sectional views of a pressure-based fluidic valve in the open and closed states.}
\label{fig:pneumatic-schema}
\end{figure}